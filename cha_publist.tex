\documentclass{article}
\usepackage[utf8]{inputenc}
\usepackage{fontspec}
\setmainfont{Arial}
\usepackage{a4wide}
\usepackage{fancyhdr}
\pagestyle{fancy}
\fancyhf{}
\rhead{Stergios Chatzikyriakidis}
\lhead{List of Publications}
%\rfoot{Page \thepage}\begin{document}
\rfoot{}

\begin{document}

\section*{Stergios Chatzikyriakidis}


\begin{enumerate}

\item Bernardy, JP., Blanck, R., \textbf{Chatzikyriakidis, S.}, Lappin, S. and Maskharashvili A. 2019. Bayesian Inference Semantics: A Modelling System and A Test Suite. Proceedings of the Eighth Joint Conference on Lexical and Computational Semantics. 

\item Bernardy, JP., Blanck, R.,  \textbf{Chatzikyriakidis, S.} and Lappin S. 2018. A compositional Bayesian semantics for natural language. Proceedings of the workshop on Language, Cognition and Computational Models, COLING 2018.


\item \textbf{Chatzikyriakidis, S.} and Luo, Z. (eds), 2017. \textit{Modern Perspectives in Type-Theoretical Semantics: The State of the Art}. Springer.


\item \textbf{Chatzikyriakidis, S.,}  Cooper, R.,  Dobnik, S. and  Larsson, S. 2017. An overview of Natural Language Inference Data Collection: The way forward?. \textit{Proceedings of the workshop on Computing Natural Language Inference}(CONLI), IWCS2017.

\item \textbf{Chatzikyriakidis, S.} and Luo, Z. 2017. Adjectival/adverbial modification: The view from modern type theories. \textit{Journal of Language, Logic and Information}, Volume 26 (1), 45–88.


\item \textbf{Chatzikyriakidis, S.} and  Luo, Z. (to appear). \textit{Formal Semantics in Modern Type Theories}. IST-Wiley Science Publishing Ltd. 



\item Kempson R., Cann R., Gregoromichelaki E. and \textbf{Chatzikyriakidis, S.} 2016. Language as Mechanisms for Interaction. \textit{Theoretical Linguistics} 42.


\item Bizzoni, Y., \textbf{Chatzikyriakidis, S.} and Ghanimifard, M. 2017. "Deep Learning": Detecting Metaphoricity in Adjective-Noun Pairs. In \textit{the proceedings of EMNLP2017}, workshop on Stylistic Variation. ACL Anthology.

\item Bernardy, J.P and \textbf{Chatzikyriakidis S.} 2017. A Type-Theoretical system for the FraCaS test suite: Grammatical Framework meets Coq. In the \textit{proceedings of IWCS 2017},  ACL anthology.


\item \textbf{Chatzikyriakidis, S.} and Zhaohui L. 2013a. Adjectives in a modern type-theoretical setting. In Morrill,
G., Nederhof, J. (eds.), \textit{proceedings of Formal Grammar 2013, LNCS 8036}, 159-174.




\end{enumerate}
\newpage
\rhead{Christine Howes}
\lhead{List of Publications}
%\rfoot{Page \thepage}
\section*{Christine Howes}
\begin{enumerate}
\item Purver, M., Hough, J. \& \textbf{Howes, C.} (2018). Computational models of miscommunication phenomena. Topics in Cognitive Science. (Peer reviewed article)
\item Breitholtz, E., \textbf{Howes, C.} \& Cooper, R (2017). Incrementality all the way up. In Computing Natural Language Inference Workshop at the International Conference on Computational Semantics (IWCS). (Conference publication)
\item Dobnik, S. \& \textbf{Howes, C} (2017). Towards a computational model of frame of reference alignment in dialogue. In Proceedings of the 7th Joint Action Meeting (JAM). London, UK. (Conference publication)
\item Hård af Segerstad, Y., Kullenberg, C., Kasperowski, D. \& \textbf{Howes, C} (2017). Studying closed communities on-line: Digital methods and ethical considerations beyond informed consent and anonymity. In Zimmer, M. \& Kinder-Kurlanda, K. (editors), Internet Research Ethics for the Social Age: New Cases and Challenges. Peter Lang. (Book chapter)
\item \textbf{Howes, C.}, Healey, P. G., Panzarasa, P. \& Hills, T (2015). Adaptation and interaction in collaborative problem solving. In 14th International Pragmatics Conference. Antwerp, Belgium. (Conference publication)
\item Kempson, R., Gregoromichelaki, E. \& {Howes, C} (2015). Languages as mechanisms for interaction. In Experimental Psychology Society (EPS) workshop on language in context: an ecological turn to embodied language. (Conference publication)
\item \textbf{Howes, C.}, Purver, M. \& McCabe, R (2014). Linguistic indicators of severity and progress in online text-based therapy for depression. In Proceedings of the ACL Workshop on Computational Linguistics and Clinical Psychology: From Linguistic Signal to Clinical Reality, pages 7-16. Baltimore, ML, USA : Association for Computational Linguistics. (Conference publication)
\item \textbf{Howes, C.}, Purver, M. \& McCabe, R. (2013). Using conversation topics for predicting therapy outcomes in schizophrenia. Biomedical Informatics Insights, 6(Suppl. 1), 39-50. (Peer reviewed article)
\item \textbf{Howes, C.}, Purver, M. \& McCabe, R (2013). Investigating topic modelling for therapy dialogue analysis. In Proceedings of IWCS 2013 Workshop on Computational Semantics in Clinical Text (CSCT 2013), pages 7-16. Potsdam, Germany : Association for Computational Linguistics. (Conference publication)

\item \textbf{Howes, C.}, Purver, M., Healey, P. G., Mills, G. J. \& Gregoromichelaki, E. (2011). On incrementality in dialogue: Evidence from compound contributions. Dialogue and Discourse, 2(1), 279-311. (Peer reviewed article)
\end{enumerate}
\newpage
\rhead{Simon Dobnik}
\lhead{List of Publications}
%\rfoot{Page \thepage}
\section*{Simon Dobnik}


\begin{enumerate}

\item \textbf{Simon Dobnik}, Mehdi Ghanimifard, and John D. Kelleher. 2018. Explorations of functionally- and geometrically-biased spatial relations with neural language models. In Proceedings of the First International Workshop on Spatial Language Understanding (SpLU 2018) at NAACL-HLT 2018, pages 1–10, New Orleans, Louisiana, USA. North American Chapter of the Association for Computational Linguistics: Human Language Technologies

\item \textbf{Simon Dobnik} and John D. Kelleher. 2017. Modular mechanistic networks: On bridging mechanistic and phenomenological models with deep neural networks in natural language processing. In Simon Dobnik and Shalom Lappin, editors. 2017. CLASP Papers in Computational Linguistics: Proceedings of the Conference on Logic and Machine Learning in Natural Language (LaML 2017), Gothenburg, 12–13 June, volume 1. pages 1–11

\item \textbf{Simon Dobnik} and Amelie Åstbom. 2017. (Perceptual) grounding as interaction. In Proceedings of Saardial – Semdial 2017: The 21st Workshop on the Semantics and Pragmatics of Dialogue, pages 17–26, Saarbrücken, Germany

\item \textbf{Simon Dobnik} and Robin Cooper. 2017. Interfacing language, spatial perception and cognition in Type Theory with Records. Journal of Language Modelling, 5(2):1–29.

\item Mehdi Ghanimifard and \textbf{Simon Dobnik}. 2017. Learning to compose spatial relations with grounded neural language models. In Proceedings of IWCS 2017: 12th International Conference on Computational Semantics, pages 1–12, Montpellier, France. Association for Computational Linguistics

\item \textbf{Simon Dobnik} and John D. Kelleher. 2016. A model for attention-driven judgements in type theory with records. In JerSem: The 20th Workshop on the Semantics and Pragmatics of Dialogue, volume 20, pages 25–34, New Brunswick, NJ USA.

\item Robin Cooper, \textbf{Simon Dobnik}, Shalom Lappin, and Staffan Larsson. 2015. Probabilistic type theory and natural language semantics. Linguistic Issues in Language Technology (LiLT), 10(4):1–43.

\item \textbf{Simon Dobnik} and John D. Kelleher. 2014. Exploration of functional semantics of prepositions from corpora of descriptions of visual scenes. In Proceedings of the Third V\&L Net Workshop on Vision and Language, pages 33–37, Dublin, Ireland. Dublin City University and the Association for Computational Linguistics.

\item \textbf{Simon Dobnik} and John D. Kelleher. 2013. Towards an automatic identification of functional and geometric spatial prepositions. In Proceedings of PRE-CogSsci 2013: Production of referring expressions – bridging the gap between cognitive and computational approaches to reference, pages 1–6, Berlin, Germany.

\item \textbf{Simon Dobnik}. 2009. Teaching mobile robots to use spatial words. PhD thesis, University of Oxford: Faculty of Linguistics, Philology and Phonetics and The Queen’s College, Oxford, United Kingdom, September 4.

\end{enumerate}

\newpage
\rhead{Robin Cooper}
\lhead{List of Publications}
%\rfoot{Page \thepage}
\section*{Robin Cooper}



\begin{enumerate}
\item  \textbf{Cooper, Robin}, Simon Dobnik, Shalom Lappin and Staffan Larsson (2015) Probabilistic Type Theory and Natural Language Semantics, Linguistic Issues in Language Technology, Vol. 10 (Peer-reviewed original article)

\item Dobnik, Simon and \textbf{Robin Cooper} (2017) Interfacing language, spatial perception and cognition in Type Theory with Records, Journal of Language Modelling, Vol. 5 No. 2, pp. 272–301 (Peer-reviewed original article)

\item Dobnik, S. ; \textbf{Cooper, R.} ; Larsson, S. (2013). Modelling Language, Action, and Perception in Type Theory with Records. Constraint Solving and Language Processing: 7th International Workshop, CSLP 2012, Orle\'ans, France, September 13-14, 2012, Revised Selected Papers / Editors: Denys Duchier, Yannick Parmentier . Berlin Heidelberg: Springer-Verlag. 70–91. ISBN/ISSN: 978-3-642-41577-7 (Conference contribution)

\item Dobnik, Simon; \textbf{Cooper, Robin}; Larsson, Staffan (2015) A formal semantic model for spatial descriptions. Proceeds of the Workshop Formal Semantics Meets Cognitive Semantics, January 22-23 2015, Radboud University Nijmegen, The Netherlands.  (Conference contribution)

\item Ginzburg, Jonathan, \textbf{Robin Cooper} and Tim Fernando (2014) Propositions, Questions, and Adjectives: a rich type theoretic approach, in Proceedings of the Workshop on Type Theory and Natural Language Semantics (TTNLS), 14th Conference of the European Chapter of the Association for Computational Linguistics, ed. by Cooper, Robin, Simon Dobnik, 
Shalom Lappin and Staffan Larsson, Association for Computational Linguistics, pp. 89–96  (Conference contribution)


\item \textbf{Cooper, Robin)} (2017) Neural TTR and possibilities for learning in S. Dobnik and S. Lappin, eds., CLASP Papers in Computational Linguistics: Proceedings of the Conference on Logic and Machine Learning in Natural Language (LaML 2017), Gothenburg, 12–13 June, vol. 1 of CLASP Papers in Computational Lin- guistics. Department of Philosophy, Linguistics and Theory of Science (FLOV), University of Gothenburg, Gothenburg, Sweden: CLASP, Centre for Language and Studies in Probability.  (Conference contribution)


\item \textbf{Cooper, Robin} (2012) Type Theory and Semantics in Flux, in Handbook of Philosophy of Science, Volume 14,
Philosophy of Linguistics, ed. by Ruth Kempson, Tim Fernando and Nicholas Asher, Elsevier. (Book chapter)

\item \textbf{Cooper, Robin} (2017) Adapting Type Theory with Records for Natural Language Semantics, in Modern Per- spectives in Type-Theoretical Semantics, ed. by Stergios Chatzikyriakidis and Zhaohui Luo, eds., Studies in Linguistics and Philosophy, 98, Springer, pp. 71–9 (Book chapter)

\item \textbf{Cooper, Robin} and Jonathan Ginzburg (2015) Type Theory with Records for Natural Language Semantics, in Handbook of Contemporary Semantic Theory (second edition), ed. by Shalom Lappin and Chris Fox, Wiley-Blackwell, pp. 375–407 (Book chapter)

%\item \textbf{Cooper, Robin} (2016) Frames as records, in Formal Grammar: 20th and 21st International Conferences, FG 2015, Barcelona, Spain, August 2015, Revised Selected Papers. FG 2016, Bozen, Italy, August 2016, Pro- ceedings, ed. by A. Foret, G. Morrill, R. Muskens, R. Osswald, S. Pogodalla, Theoretical Computer Science and General Issues, Vol. 9804, Springer, pp. 3–18 (Book chapter)

\item Stergios Chatzikyriakidis and \textbf{Robin Cooper} (2018) Type Theory for Natural Language Semantics, in Oxford Research Encyclopedia of Linguistics,\\ DOI:10.1093/acrefore/9780199384655.013.329 (Book chapter)
\end{enumerate}

\newpage
\rhead{Rasmus Blanck}
\lhead{List of Publications}
%\rfoot{Page \thepage}
\section*{Rasmus Blanck}


\begin{enumerate}

\item Blanck, R. \& Enayat, A. (2017). Marginalia on a theorem of Woodin. The Journal of Symbolic Logic, 82(1), 359-374.  (Peer-reviewed original article)


\item Blanck, R. (2017). Contributions to the Metamathematics of Arithmetic: Fixed Points, Independence, and Flexibility, Acta Philosophica Gothoburgensia vol. 30, Acta Universitatis Gothoburgensis, University of Gothenburg. (Book)


\item Blanck, R. (2016). Flexibility in Fragments of Peano Arithmetic, in Studies in Weak Arithmetics, Volume 3, eds. Patrick Cégielski, Ali Enayat, Roman Kossak, 1-20, CSLI Lecture Notes 217, Stanford. (Book chapter)

\item Blanck, R. (2011). Metamathematical fixed points, Philosophical communications, Red series, vol. 41, University of Gothenburg. (Book)



\item Blanck, R. (2014). Two consequences of Kripke’s lemma, in Idées fixes, ed. Martin Kaså, Philosophical communications, Web series, vol. 61, University of Gothenburg. (Other publication including popular science books/presentation)
\end{enumerate}


\end{document}
